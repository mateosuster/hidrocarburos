%++++++++++++++++++++++++++++++++++++++++
% Don't modify this section unless you know what you're doing!
\documentclass[letterpaper,11pt, spanish]{scrartcl}
\usepackage{natbib}
\bibliographystyle{unsrtnat}
\usepackage{tabularx} % extra features for tabular environment
\usepackage{amsmath}  % improve math presentation
\usepackage{graphicx} % takes care of graphic including machinery
\usepackage[margin=1in,letterpaper]{geometry} % decreases margins
%\usepackage{cite} % takes care of citations
\usepackage[final]{hyperref} % adds hyper links inside the generated pdf file
\hypersetup{
	colorlinks=true,       % false: boxed links; true: colored links
	linkcolor=blue,        % color of internal links
	citecolor=blue,        % color of links to bibliography
	filecolor=magenta,     % color of file links
	urlcolor=blue         
}

\usepackage[spanish]{babel}
\selectlanguage{spanish}
\usepackage[utf8]{inputenc}
\usepackage{enumitem}

\newcommand{\SubItem}[1]{
    {\setlength\itemindent{15pt} \item[-] #1}
}
%++++++++++++++++++++++++++++++++++++++++


\begin{document}

\title{Renta de la tierra hidrocarburífera argentina}
\subtitle{Metodología y fuentes}
\author{MS y JK}
\date{Junio 2021}
\maketitle

\begin{abstract}
....
\end{abstract}


\section{Fuentes recopiladas}
\subsection{Producción}

\begin{itemize}
    \item \hyperlink{https://www.argentina.gob.ar/economia/politicaeconomica/regionalysectorial/informesproductivos}{Ministerio de Hacienda, Informes Sectoriales (1998-hoy)}.
    \item \hyperlink{http://datos.minem.gob.ar/dataset/regalias-de-petroleo-crudo-gas-natural-glp-gasolina-y-condensado}{Secretaría de Energía - Regalias (1998-hoy)}
    \item \hyperlink{http://datos.minem.gob.ar/dataset/produccion-de-petroleo-y-gas-tablas-dinamicas}{Secretaría de Energía - SESCO (1993-hoy) }
    \item \hyperlink{http://datos.minem.gob.ar/dataset/anuarios-de-combustibles-1950-1999}{Anuario de combustibles (1911-1992)}
    \item Kozulj y Pistonesi. Revista  del Instituto de Economía Energética (IDEE) - Fundación Bariloche  (1970 - 1988)
    \item \hyperlink{https://www.eia.gov/international/data/country/ARG/petroleum-and-other-liquids/annual-petroleum-and-other-liquids-production?pd=5&p=0000000000000000000000000000000000vg&u=0&f=A&v=mapbubble&a=-&i=none&vo=value&&t=C&g=none&l=249--6&s=94694400000&e=1546300800000)}{EIA  (1980 - 2019)}
\end{itemize}

\subsection{Precios del mercado interno}

\begin{itemize}
    \item \hyperlink{https://www.argentina.gob.ar/economia/politicaeconomica/regionalysectorial/informesproductivos}{Base Ministerio de Hacienda, Informes
Sectoriales}.
    \item \hyperlink{http://datos.minem.gob.ar/dataset/regalias-de-petroleo-crudo-gas-natural-glp-gasolina-y-condensado}{Secretaría de Energía - Regalias (1998-hoy)}
    \item \hyperlink{http://datos.minem.gob.ar/dataset/produccion-de-petroleo-y-gas-tablas-dinamicas}{Secretaría de Energía - SESCO (1993-hoy)}. Por ausencia de información, el precio de base Regalias entre 1993 y 1998 es el precio total (es decir, es un ponderado que incluye también el precio del mercado externo). En los años posteriores, dicho precio sí corresponde al del mercado interno. 
    \item \hyperlink{http://datos.minem.gob.ar/dataset/anuarios-de-combustibles-1950-1999}{Anuario de combustibles (1911-1992)}
    \item Kozulj y Pistonesi. Revista  del Instituto de Economía Energética (IDEE) - Fundación Bariloche  (1970 - 1988). Precio oficial interno de la cuenca neuquina a tasa de cambio oficial. Las fuentes utilizadas de esta revista son Secretaria de Energía, YPF, Gas del Estado, Boletin Informativo de Techint y series propias de IDEE
\end{itemize}

\subsection{Precios del mercado interno}

\begin{itemize}
    \item \hyperlink{https://www.argentina.gob.ar/economia/politicaeconomica/regionalysectorial/informesproductivos}{Base Ministerio de Hacienda, Informes
Sectoriales}.
    \item \hyperlink{http://datos.minem.gob.ar/dataset/regalias-de-petroleo-crudo-gas-natural-glp-gasolina-y-condensado}{Secretaría de Energía - Regalias (1998-hoy)}
    \item \hyperlink{http://datos.minem.gob.ar/dataset/produccion-de-petroleo-y-gas-tablas-dinamicas}{Secretaría de Energía - SESCO (1993-hoy)}. Por ausencia de información, el precio de base Regalias entre 1993 y 1998 es el precio total (es decir, es un ponderado que incluye también el precio del mercado externo). En los años posteriores, dicho precio sí corresponde al del mercado interno. 
    \item \hyperlink{http://datos.minem.gob.ar/dataset/anuarios-de-combustibles-1950-1999}{Anuario de combustibles (1911-1992)}
    \item Kozulj y Pistonesi. Revista  del Instituto de Economía Energética (IDEE) - Fundación Bariloche  (1970 - 1988). Precio oficial interno de la cuenca neuquina a tasa de cambio oficial. Las fuentes utilizadas de esta revista son Secretaria de Energía, YPF, Gas del Estado, Boletin Informativo de Techint y series propias de IDEE
\end{itemize}

\subsection{Precios de exportación y referencia del mercado mundial}
Precios de exportación desde Argentina, benchmarks y precios del mercado de EEUU (sólo gas, internos y de exportación e importación)

\begin{itemize}
    \item \hyperlink{http://datos.minem.gob.ar/dataset/precio-de-exportacion-de-petroleo-crudo}{Ministerio de Economia (MECON)}.
    \item \hyperlink{http://datos.minem.gob.ar/dataset/regalias-de-petroleo-crudo-gas-natural-glp-gasolina-y-condensado}{Secretaría de Energía - Regalias (1998-hoy)}
    \item \hyperlink{https://www.eia.gov/dnav/pet/hist/RBRTED.htm}{EIA - Brent (1987-hoy)} 
    \item \hyperlink{https://www.eia.gov/dnav/pet/hist/rwtcD.htm}{EIA - WTI (1986-hoy)}
    \item \hyperlink{(https://www.eia.gov/dnav/ng/hist/rngwhhdd.htm}{EIA - Henry Hub (1997-hoy)}
    \item \hyperlink{(https://www.eia.gov/dnav/ng/hist/n9190us3A.htm}{EIA - Precio gas boca de pozo de EEUU}
    \item \hyperlink{https://inflationdata.com/articles/inflation-adjusted-prices/historical-crude-oil-prices-table/)}{Inflation Data (1946-hoy)}
    \item \hyperlink{https://ec.europa.eu/eurostat/web/energy/data/database}{Eurostat}
    \item \hyperlink{https://comtrade.un.org/data/}{UN Comtrade}. Precio de Exportación e Importación de gas (Bolivia) y crudo (promedio mundial)
    \item Kozulj y Pistonesi. Revista  del Instituto de Economía Energética (IDEE) - Fundación Bariloche  (1970 - 1988). Precio de Importación de gas natural desde Bolivia (1970-1988)    
    \item \hyperlink{https://comtrade.un.org/data/}{UN Comtrade}
\end{itemize}

\subsection{Exportaciones e Importaciones}


\begin{itemize}
    \item \hyperlink{https://datos.gob.ar/dataset/energia-refinacion-comercializacion-petroleo-gas-derivados-tablas-dinamicas}{SESCO Downstream}.
    \item \hyperlink{}{INDEC}
    \item \hyperlink{}{MECON}
    \item \hyperlink{}{UNComtrade}

\end{itemize}

\subsection{Empleo y remuneraciones}

\begin{itemize}
    \item \hyperlink{}{Base Minería e Hidrocarburos de las Cuentas Nacionales (1996-2013)}
    \item \hyperlink{http://www.trabajo.gob.ar/estadisticas/oede/estadisticasnacionales.asp}{Ministerio de Trabajo, Empleo y Seguridad Social - Observatorio de Empleo y Dinámica Empresarial (OEDE) (1996-2019)}
\end{itemize}


\subsection{Activos}
\begin{itemize}
    \item Balances de Bolsar (unicamente empresas que cotizan en la bolsa de valores argentina)
    \item AFIP - Anuario Estadístico. Nota: "\_c" significa casos presentados de la variable correspondiente
    \item Memorias de YPF
\end{itemize}

\subsection{Regalías}
\begin{itemize}
    \item \hyperlink{http://datos.minem.gob.ar/dataset/regalias-de-petroleo-crudo-gas-natural-glp-gasolina-y-condensado}{Secretaría de Energía - Regalias (1998-hoy)}
\end{itemize}

\section{Fuentes seleccionadas para la construcción de series}

\subsection{Producción de Crudo} 
\begin{itemize}
    \item 1911 a 1992: Anuario de combustibles
    \item 1993 - actualidad: SESCO Downstream
\end{itemize} 
 
\subsection{Exportaciones de Crudo}
\begin{itemize}
    \item 1962 a 1993: UN Comtrade (clasificación SITC as reported)
    \item 1994 en adelante: SESCO Downstream
    \item Los datos faltantes de SESCO se completaron con MECON. Se presentan faltantes para los años 1965, 1970-74, 1976-78, 1980-84 
\end{itemize}
  
\subsection{Precio Mercado Interno de Crudo}
\begin{itemize}
    \item 1963 a 1965: Kozulj y Pistonesi - Revista IDEE ajustado con el índice del precio del Anuario de YPF 
    \item 1989 a 1991: Anuario de YPF
    \item 1966 a 1988: Kozulj y Pistonesi - Revista IDEE
    \item 1992: MECON ajustado con la variación del Índice de precios internos al por mayor (IPIM)
    \item 1993 en adelante: MECON
\end{itemize}
   
\subsection{Precio Mercado Externo Crudo}
\begin{itemize}
    \item entre 1962 y 1992: precio de exportación argentina de UN Comtrade
    \item entre 2002 y 2003: precio de exportación de Argentina de UN Comtrade (Clasificación HS as reported) 
    \item 2014 en adelante: precio de exportación argentina de Secretaría de Energía (Regalías)
    \item Valores faltantes previos a 1992: Brent  (Fuente: Inflation Data)
\end{itemize}

\subsection{Producción de Gas}
\begin{itemize}
    \item 1911 a 1992: Anuario de combustibles
    \item 1993 en adelante: SESCO Downstream
\end{itemize}


\subsection{Exportaciones de gas}
\begin{itemize}
    \item 1962 a 1996: UN Comtrade,
   \item 1997 en adelante: SESCO Downstream
  \item Datos faltantes para los años 1999 y 2000
\end{itemize}


\subsection{Precio Mercado Interno de gas}
\begin{itemize}
    \item  1963 - 1969 y 1989 - 1992: Anuario de YPF
  \item  1970 - 1988:  Kozulj y Pistonesi - Revista IDEE
\item  1993 en adelante: Secretaría de Energía - Base Regalías
\end{itemize}
  
 \subsection{Precio Mercado Externo}
\begin{itemize}
    \item Años 1964 y 1965: Precio de importación de gas proveniente de Bolivia hacia Argentina de UN Comtrade
  \item 1966 en adelante: Precio de exportacion de gas desde Bolivia a Argentina de UN Comtrade
  \item Datos faltantes para los años previos a 1963 y período 1968-1971
\end{itemize}
 
 
\subsection{Salario y empleo}
\begin{itemize}
    \item  1960-1996: estimación a partir de aplicación del coeficiente de proporción de la masa salarial sobre el VBP
  \item 1996 en adelante: Ministerio de Trabajo, Empleo y Seguridad Social - Observatorio de Empleo y Dinámica Empresarial (OEDE)
\end{itemize}

\subsection{Consumo de capital fijo}
\begin{itemize}
    \item Coeficiente de depreciación estimado a partir de Estados Contables de YPF (1998-2018)
\end{itemize}




\section{Criterios de cómputo}


\subsection{Valor total de la producción }
Se presentan a continuación distintas estimaciones sobre la magnitud de riqueza presente en el sector hidrocarburífero: Valor Bruto y Agregado de Producción (VBP y VA), Consumo Interemedio (CI), Masa Salarial (MS) y Excedente Bruto de Explotación (EBE). El VBP surge de la valuación de la producción a sus precios correspondientes. El VA resulta de la diferencia entre el VBP y CI, el cual puede surgir originalmente de esta resta o a partir del coeficiente técnico de la Matriz Insumo Producto (MIP). El EBE constituye la plusvalía (PV) total de la rama, es decir, la suma de la renta de la tierra (RT) más la ganancia normal (Gnorm), y se obtiene luego de descontar la MS y los impuestos específicos (Imp) del VA. En todos los casos que se presentarán a continuación, los Imp se calcularon a partir de aplicar sobre el VBP un coeficiente  resultante del peso de los impuestos promedio de la  MIP de 1997. Lo mismo pasa con la estimación de la depreciación de capital o consumo de capital fijo (ConsKfijo), que se obtiene a partir de aplicar la tasa de depreciación promedio resultante de los balances de YPF (1998 - 2018) sobre el total de Propiedad, Planta y Equipo (PPyE) de la rama. Esta partida se aplica para obtener el Excedente Neto de Explotación. Se presentan distintas estimaciones para el VBP, CI y MS, que luego se observarán en la formulación matemática.

\begin{itemize}
    \item CCNN: Valores contables oficiales de las Cuentas Nacionales (sólo disponible para el período 2004-2012). Dado que se presentan series de VBP y VA, se pudo estimar el CI como la diferencia de dichas cuentas. Se procedió a separar el VBP a partir del peso del VBP de los servicios de apoyo a la extracción sobre el VBP de extracción de petróleo y gas, presente en el Cuadro de Utilización de Oferta (COU) de 2004 de INDEC. Se descontó esta proporción (resultante del 10,7\%) del VBP total para obtener un VBP sólo de extracción. Dado que se posee información del salario promedio del sector y el empleo, se pudo obtener la MS resultante, tanto para el total del sector (extracción y servicios relacionados) como para sólo extracción. A partir de estos datos se elaboró un coeficiente que refleja el promedio de la proporción de la MS sobre el VBP que se utilizará en cálculos posteriores de MS total y MS de extracción. 
    \item Estimación propia con criterio CCNN: Estimación propia de los valores contables a partir de las fuentes recopiladas, siguiendo los criterios de las Cuentas Nacionales. Es decir, para obtener el VBP se valua la producción destinada al mercado interno (resultante a partir de la diferencia entre producción y exportaciones) con los precios internos y las exportaciones con los precios de exportación, valuados tipo de cambio comercial (TCC). Se procedió también a separar el VBP de extracción neto de los servicios tal como se explicó anteriormente. El CI se estimó a partir de aplicar el coeficiente técnico (ratio CI/VBP) resultante de la MIP de 1997 (equivalente a 0.272). De igual manera, para calcular la masa salarial se aplicó el coeficiente de MS mencionado anteriormente. Finalmente, como se mencionó anteriormente, el VA el EBE se calcularon a partir de las diferencias mencionadas anteriormente.
    \item Empalme CCNN: Estimación que toma los valores oficiales de las cuentas nacionales para el período donde se presentan datos (2004 -2012) y que imputa los valores faltantes por medio de la evolución del índice del VBP propio con criterio CCNN explicado anteriormente. Asimismo, se utilizaron los valores oficiales de la MS cuando se encontraba disponible el dato (1996-2018), mientras que se utilizó el valor propio estimado con criterio de las CCNN para los restantes años. \item Criterio Propio: Estimación propia que refleja con mayor precisión el valor de la riqueza total presente en el sector. El VBP se obtiene a paritr de valuar la totalidad de la producción a los precios externos o de referencia internacional y con el tipo de cambio de paridad (TCP), que mide la capacidad real de compra de la moneda nacional. Sin embargo, como el CI constituye intercambios de mercancías realizadas en el ámbito nacional, dicha partida se obtiene a partir de los valores obtenidos en la serie de empalme CCNN. De igual manera, se utilizó la MS resultante de esta última estimación. 
    
\subsubsection{Formulación matemática}

Valor Bruto de Producción total, estimación con criterio propio
$$VBP_{propia} = (Pext_{petroleo} * Q_{petróleo} + Pext_{gas} * Q_{gas}) * TCP$$

Donde: 
\begin{itemize}
\item $VBP_{propia}$ = Valor Bruto de la Producción total, estimación propia
\item $Pext_{petróleo}$ = Precio de exportación o referencia internacional del petróleo crudo (según corresponda)
\item $Pext_{gas}$ = Precio de exportación o referencia internacional del gas natural (según corresponda)
\item $Q_{petróleo}$ = Cantidades producidas totales de petróleo crudo
\item $Q_{gas}$ = Cantidades producidas totales de gas natural
\item $TCP$ = Tipo de Cambio de Paridad \\ \\
\end{itemize}


Valor Bruto de Producción total, estimación con criterio CCNN
$$VBP_{CCNN} = (Pint_{petróleo} * QMInt_{petróleo} + Pext_{petróleo} * Expo_{petróleo} +$$
$$Pint_{gas} * QMInt_{gas} + Pext_{gas} * Expo_{gas})* TCC$$

Donde:
\begin{itemize}

\item $VBP_{CCNN}$ = Valor Bruto de la Producción total,  estimación propia con criterio de las CCNN
\item $Pint_{petróleo}$ = Precio mercado interno del petróleo crudo
\item $Pint_{gas}$ = Precio mercado interno del gas natural
\item $QMInt_{petróleo}$ = cantidades vendidas al mercado interno de petróleo crudo
\item $QMInt_{gas}$ = cantidades vendidas al mercado interno del gas natural
\item $Expo_{petróleo}$ = exportaciones de petróleo crudo
\item $Expo_{gas}$ = exportaciones de gas natural
\item $TCC$ = Tipo de Cambio Comercial\\ \\

\end{itemize}

Valor Bruto de Producción extracción, estimación con criterio CCNN

$$VBP\_extr_{CCNN} = VBP_{CCNN} * (1-prop\_servicios)$$

Donde:
\begin{itemize}
\item $VBP\_extr_{CCNN}$ = Valor Bruto de la Producción extracción,  estimación propia con criterio de las CCNN
\item $prop\_extr$ = proporción del VBP de servicios de apoyo sobre VBP de extracción de petróleo y gas\\ \\
\end{itemize}


Proporción de los servicios de apoyo sobre la extracción de petróleo y gas
$$prop\_servicios = \frac{VBP\_servicios_{COU}}{VBP\_extr_{COU}}$$
Donde:
\begin{itemize}
\item $VBP\_servicios_{COU}$ = VBP de servicios de apoyo del Cuadro de Utilización de Oferta
\item $VBP\_extracción_{COU}$ = VBP de extracción de petróleo y gas del Cuadro de Utilización de Oferta\\ \\
\end{itemize}


Consumo Intermedio, valores oficiales de las CCNN
$$ CI_{CCNN} = VBP_{CCNN} -  VA_{CCNN} $$

Donde:
\begin{itemize}
\item $CI_{CCNN}$ = Consumo Intermedio total, estimación propia
\item $VA_{CCNN}$ = Valor Agregado, estimación de las CCNN \\ \\
\end{itemize}


Consumo Intermedio, estimación con criterio CCNN
$$ CI_{CCNN} = VBP_{CCNN} *  Coef\_tec $$
Donde:
\begin{itemize}
\item $Coef\_tec$ = Coeficiente técnico de Matriz Insumo Producto\\ \\
\end{itemize}


Consumo Intermedio de extracción, estimación con criterio CCNN
$$ CI\_extr_{CCNN} = VBP\_extr_{CCNN} *  Coef\_tec $$
Donde:
\begin{itemize}
\item $ CI\_extr_{CCNN}$ = Consumo Intermedio de extracción, estimación criterio CCNN\\ \\
\end{itemize}


Masa Salarial, valores oficiales de las CCNN
$$MS = W * Emp * 13$$
Donde: 
\begin{itemize}
\item $MS$ = Masa Salarial 
\item $W$ =  Salario anual promedio
\item $Emp$ = Empleo\\ \\
\end{itemize}


Masa Salarial, estimación con criterio CCNN
$$MS =  VBP_{CCNN} *  Coef\_MS$$

Donde: 
\begin{itemize}
\item $Coef\_MS$ = Coeficiente de la proporción de MS sobre VBP\\ \\
\end{itemize}


Masa Salarial de extracción, estimación con criterio CCNN
$$MS\_extr =  VBP\_extr_{CCNN} *  Coef\_MS$$

Donde: 
\begin{itemize}
\item $MS\_extr$ = Masa salarial del sector extracción\\ \\
\end{itemize}


Valor agregado, estimación criterio CCNN
$$VA_{CCNN}  = VBP_{CCNN} – CI_{CCNN} $$
Donde:
\begin{itemize}
\item $VA_{CCNN}$ = Valor Agregado, estimación propia con criterio CCNN\\ \\
\end{itemize}



Valor agregado de extracción, estimación con criterio CCNN
$$VA\_extr_{CCNN}  = VBP\_extr_{CCNN} – CI\_extr_{CCNN} $$

Donde:
\begin{itemize}
\item $VA\_extr_{CCNN}$ = Valor Agregado de extracción, estimación propia con criterio CCNN \\ \\
\end{itemize}

Valor agregado, estimación con criterio propio
$$VA_{propia}  = VBP_{propia} – CI\_extr_{CCNN} $$

Donde:
\begin{itemize}
\item $VA_{propia}$ = Valor Agregado, estimación con criterio propio \\ \\
\end{itemize}

Excedente Bruto de Explotación, estimación con criterio CCNN
$$EBE_{CCNN}  = VA_{CCNN} – MS   $$\\


Excedente Bruto de Explotación de extracción, estimación con criterio CCNN
$$EBE\_extr_{CCNN}  = VA\_extr_{CCNN} – MS\_extr   $$ \\

Excedente Bruto de Explotación, estimación con criterio propio
$$EBE_{propia}  = VA_{propia} – MS\_extr   $$\\ 


Donde:
\begin{itemize}
\item $EBE$ = Excedente Bruto de Explotación
\item $CI\_extr_{CCNN}$ = Consumo intermedio del sector extracción, estimación con criterio CCNN\\ \\
\end{itemize}



Consumo de Capital Fijo, estimación con criterio propio
$$ConKfijo = PPyE *  prom(\frac{Dep}{PPyE}) $$

Donde: 
\begin{itemize}
\item $ConKfijo$ = Consumo de Capital Fijo
\item $PPyE$= Propiedad, Planta y Equipo neta
\item $prom(\frac{Dep}{PPyE}) $ =  tasa de depreciación promedio
\item $Dep$ = Depreciaciones (cuenta gastos por naturaleza)\\ \\
\end{itemize}



Plusvalía (Excedente Neto de Explotación), estimación con criterio propio
$$PV_{propia} = VA_{propia} - ConKfijo - Imp $$


Donde:
\begin{itemize}
    \item 
\end{itemize}
\item $PV_{propia}$ = Plusvalía (*Excedente Neto de Explotación*)
\item $Imp$ = Impuestos normales\\ \\
\end{itemize}


\end{document}



\subsection{Calculo del Capital Total Adelantado (KTA)}
\begin{itemize}
    \item Bolsar: Es equivalente a la suma de Propiedad, Planta y Equipos Neta (descontando los terrenos y obras en curso) y los Inventarios. Cuando los datos lo habilitan, se le agregó los salarios adelantados (salarios y cargas consumidos sobre rotación). Luego, cuando no se presentaron datos de Propiedad, Planta y Equipos, se tomó el activo no corriente.
    \item AFIP: Es equivalente a la suma de Bienes de Uso, Bienes de Cambio, Inventarios y Disponibilidades.
    \item Memoria YPF: Suma de Bienes de Uso y Bienes de Cambio
\end{itemize}



SEGUIR ACA
Tasa de Ganancia


$$TG_{hidrocarburos} = \frac{PV_{hidrocarburífera}}{KTA_{hidrocarburífero}}$$

## Renta apropiada por las empresas de la rama

La renta apropiada por las empresas de la rama se calcula por medio del diferencial de tasas de ganancia entre el sector hidrocarburífero que surge a partir de los balances y la rentabilidad normal de la economía.

$$Renta\_empresas = KTA_{hidrocarburífero} * (TG_{hidrocarburífera} - TG_{referencia})$$
$$TG_{hidrocarburífera} = \frac{Gcia_{hidrocarburos}}{KTA_{hidrocarburífero}}$$ 
Por lo cual, la renta apropiada por las empresas de la rama sería equivalente a:

$$Renta\_empresas = Gcia_{hidrocarburos} - KTA_{hidrocarburífero} *  TG_{referencia}$$

Donde:

* $KTA_{hidrocarburífero}$ = Stock de capital adelantado de empresas hidrocarburíferas
* $TG_{hidrocarburífera}$ = Tasa de ganancia de empresas hidrocarburíferas
* $TG_{referencia}$ = Tasa de ganancia de referencia
* $Gcia_{hidrocarburos}$ = Ganancia de empresas hidrocarburíferas

# Renta por el diferencial de precios entre el mercado interno y las referencias internacionales 

Renta apropiada mediante el abaratamiento en el consumo interno por efecto del diferencial de precios interno/externo, sobrevaluación de la moneda y retenciones a la exportación


### Criterio de cómputo de JK (variable 'renta_dif_precios')

$$RDP= ProdInt * Pext * TCP - ProdInt * PMI * TCC$$

Donde:

* $RDP$ = Renta apropiada por efecto diferencial de precios interno/externo y sobrevaluación
* $MdoInt$ = Producción destinada al Mercado Interno: Producción - Exportaciones - Existencias (barriles de petróleo ó MMBTU)
* $Pext$ = Precio de referencia del mercado externo (USD)
* $PMI$ = Precio de venta del mercado interno (USD)
* $TCP$ = Tipo de Cambio de Paridad
* $TCC$ = Tipo de Cambio Comercial

# Renta apropiada por sobrevaluación cambiaria


$$Rsobrevaluacion = Expo * Pext * (TCP - TCC)$$ 

Donde:
* $Rsobrevaluacion$ = Renta apropiada por exportaciones con tipo de cambio sobrevaluado
* $Expo$ = Exportaciones  (barriles de petróleo ó MMBTU)
* $Pext$ = Precio de referencia del mercado externo (USD)
* $TCP$ = Tipo de Cambio de Paridad
* $TCC$ = Tipo de Cambio Comercial

# Renta apropiada por el Estado mediante impuestos específicos

$$Rimp = RE + Reg$$

Donde:

* $Rimp$ = Renta apropiada por el Estado mediante impuestos específicos
* $RE$ = Retenciones
* $Reg$ = Regalias

# Renta Hidrocarburífera Total. En precios constantes, sobre Plusvalía Total y PBI

Existen dos caminos para llegar al monto total de renta de la tierra hidrocarburífera: uno es descontando la ganancia normal de las empresas a la plusvalía total del sector y el otro es por medio de la suma de mecanismos de apropiación.

$$Renta\_hidrocarburífera = PV_{hidrocarburífera} - Gcia\_Normal_{hidrocarburífera}$$
Donde:

* $Gcia\_Normal_{hidrocarburífera}$ = Ganancia Normal del sector hidrocarburífero
* $PV_{hidrocarburífera}$ = Plusvalía del sector hidrocarburífero  

$$Gcia\_Normal_{hidrocarburífera} = KTA_{hidrocarburífero} * TG_{referencia}$$


Donde:

* $KTA_{hidrocarburos}$ = Stock de capital adelantado del sector hidrocarburífero 
* $TG_{referencia}$ = Tasa de ganancia normal de referencia.

En este caso, seleccionamos la tasa de ganancia del sector industrial como parámetro para diferenciar la renta de la ganancia. A su vez, para el capital total adelantado de las empresas hidrocarburíferas, seleccionamos unicamente el valor resultante de la estimación de la PPyE de Bolsar a partir del indice de flujo de pozos (variable "ppye_bolsar_flujo"), por lo que le faltan los inventarios y salarios adelantados.

El cálculo de renta total hidrocarburífera que se obtiene por medio de descontar la ganancia normal a la plusvalía total del sector, debe ser igual a la renta obtenida por medio de la agregación de los distintos mecanismos de apropiación. Es decir:

$$Renta\_hidrocarburífera = Renta\_diferencial\_precios + Renta\_sobrevaluación + Renta\_empresas + Impuestos\_netos\_específicos $$
$$Impuestos\_netos\_específicos = Retenciones + Regalías - Subsidios$$

## Costos

$$Q\_total = Q_{petróleo} + Q_{gas} $$
Donde:
  
  * $Q\_total$ = Cantidades producidas de petróleo y gas en Barriles Equivalentes de Petróleo
  * $Q_{petróleo}$ = Cantidades producidas de petróleo crudo en barriles equivalentes de petróleo (BOE)
  * $Q_{gas}$ = Cantidades producidas de gas natural en barriles equivalentes de petróleo (BOE)
  

$$ Costos\_totales =  CI + MS + ConKfijo$$

Donde:

* $Costos\_totales$ = Costos totales hidrocarburíferos
* $CI$ = Consumo Intermedio, distintas estimaciones
* $MS$ = Masa Salarial, distintas estimaciones
* $ConKfijo$ = Consumo de Capital Fijo


$$Costos\_totales\_con\_Gcia = Costos\_totales + Gcia\_Normal_{hidrocarburífera} $$
Donde:

* $Costos\_totales\_con\_Gcia$ = Costos totales hidrocarburíferos con ganancia normal 
* $Gcia\_Normal_{hidrocarburífera}$ = Ganancia normal del sector hidrocarburífero

$$ Precio\_costo = \frac{Costos\_totales}{Q\_total}   $$
Donde:
* $Precio\_costo$ = Precio de costo en BOE

A partir de esto se puede calcular un costo recuperable del petróleo y del gas

$$Costo\_crudo = Q_{petróleo} * Precio\_costo$$
$$Costo\_gas = Q_{gas} * Precio\_costo$$


$$Precio\_produccion = \frac{Costos\_totales\_con\_Gcia}{Q\_total}$$

$$Precio\_vta\_potencial =  \frac{Q\_total*Pext_{petróleo} - Costos\_totales}{Q\_total} $$


Donde:
* $Precio\_produccion$= Precio de produccion
* $Precio_vta_potencial$ = Precio de venta potencial
* $Pext_{petróleo}$ = Precio de exportación/referencia internacional del petróleo crudo
